\documentclass[a4paper]{article}
\usepackage{fullpage}
\author{Ryan Day}
\title{Optimization Homework \#4}
\begin{document}
    \maketitle
    \section{Truss Optimization}
    \subsection{Scaling}
    As far as I can tell, scaling the constraints would be useful if the constraints were an order of magnitude higher than the design variables.
    The design variables are all around the same order of magnitude,, so I don't see the use of scaling there.

    \begin{table}
        \begin{tabular}[h]{c c c c c}
            & Number of function calls & Number of Iterations & Avg Time of execution & Stopping Criterion \\
            No Derivatives supplied&495&19&0.573& \\
            Forward method&991&19&0.6320& \\
            Central method&3139&39&1.1457& \\ 
            Complex method&991&19&0.932&
            
        \end{tabular}
    \end{table}
%Optimization stopped because the relative changes in all elements of x are
%less than options.StepTolerance = 1.000000e-10, and the relative maximum constraint
%violation, 0.000000e+00, is less than options.ConstraintTolerance = 1.000000e-06.

%Optimization completed: The relative first-order optimality measure, 3.928372e-08, is less than options.OptimalityTolerance = 1.000000e-06, and the relative maximum constraint violation, 0.000000e+00, is less than options.ConstraintTolerance = 1.000000e-06.

% Optimization completed: The relative first-order optimality measure, 7.607382e-07,
% is less than options.OptimalityTolerance = 1.000000e-06, and the relative maximum constraint
% violation, 0.000000e+00, is less than options.ConstraintTolerance = 1.000000e-06.

% Optimization completed: The relative first-order optimality measure, 1.512774e-09,
% is less than options.OptimalityTolerance = 1.000000e-06, and the relative maximum constraint
% violation, 0.000000e+00, is less than options.ConstraintTolerance = 1.000000e-06.

\end{document}